\chapter{Housing}
\label{cha:housing}

\section{Alcohol \& Drug Policy}

Alcohol and illegal drugs are not permitted in the house or on the property.
This includes cars parked in the driveway. 

\section{Smoking Policy}

Smoking is not permitted in the house or on the property.

\section{Firearm Policy}

Firearms are not allowed in the house or on the premises.
This includes handguns, pellet guns, paintball guns, bow and arrow, or any
other projectile weapon.

\section{Internet Usage \& Copyrighted Material}

Using the house internet to illegally download or distribute copyrighted
material is prohibited.
Additionally, the rules set forth by the Internet Service Provider also apply.

\section{Chores}

\subsection{Chore Groups}

The House Manager will divide members into small groups to complete chores.
Groups may contain members both living and not living in the house based on the
type of chore.
The House Manager will then assign chores each Wednesday and the chores shall
be completed by 7:00 PM on the following Sunday.
Chore groups may contain new members as long as there is at least one brother
for every new member present.
Potential new members shall not be involved with chores.

\subsection{Dishes}

Members shall be responsible for washing their own dishes as soon as possible
after using them.
All dishes used by a member must be washed and put away within 4 hours of being
used.
If dishes are used by a guest, the member who brought that guest into the house
will be responsible for cleaning those dishes.

\subsection{Event Cleanup}
\label{sec:event-cleanup}

All members present for an event must assist with cleanup immediately following
the event.
Examples of events include, but are not limited to: rush events, socials,
brotherhood events, and rituals.


\section{General House Usage}

\subsection{Common Space}

Common spaces are defined as any area of the house that is not being used as a
bedroom.
Members shall be responsible for cleaning up after themselves and their guests
in common spaces.

\subsection{Noise}

The House Manager will establish quiet hours during which excessive noise is
prohibited.
Examples of excessive noise include, but are not limited to: amplified music,
yelling, stomping, and moving furniture.
In addition to quiet hours, members shall be considerate of their fellow
members and neighbors at all times.


\subsection{Energy Conservation}

Members shall turn off lights and power-consuming devices when not in use.
All personal electronic devices shall be unplugged prior to long breaks.
Windows shall be closed on breaks.

\subsection{Privacy}

If a bedroom door is closed, members shall knock before entering, except in the
case of an emergency.
Members agree to allow access to the house, including their bedrooms, for
repairs, alterations, improvements, inspections, or re-supply.  Where possible,
advance notice will be given before entering bedrooms.

\subsection{Roof Usage}

At no time shall any member or guest be permitted on the roof.

\subsection{Porch Usage}

At no time shall the porch capacities be exceeded.

\subsection{Kitchen Usage}

If the House Manager deems it necessary, he may create groups for using the
kitchen at certain times to avoid overcrowding.

\section{Safety \& Security}

\subsection{Locking and Management of Keys}

The house shall be kept locked at all times, unless the house manager
determines that the doors may be left unlocked.
Windows on the first floor shall also be kept locked when not in use.
Members shall not give their keys to anyone else under any circumstances,
except to return them at the end of the year.
Any member that loses his key shall report the loss to the House Manager
immediately.

\subsection{Guest Policy and Overnight Policy}

Guests must follow all house rules.
Members are responsible for any actions of their guests.
The House Manager reserves the right to remove unruly guests from the house.
Members may not house any guest overnight for more than fourteen consecutive
days.
Members shall inform the House Manager of any overnight guests.

\subsection{Room Inspections}

The Housing Corporation and House Manager will oversee a check in / check out
procedure in June and August.  This will include a room inspection checklist,
where the resident, House Manager, and Housing Corporation representative will
mutually agree on the condition of the room and contents and document any
damage to the facility.

\subsection{Fire Safety and Storage of Flammables}

All objects shall be kept away from all heaters and furnaces.
Large volumes of flammables shall not be stored in the house.
If necessary, paint may be stored in the garage.
Hanging or attaching any material to sprinklers and sprinkler pipes is strictly
prohibited.
Furthermore, hanging anything above doorways is prohibited.

\section{Parking}

\subsection{Resident Parking}
\label{sec:resident-parking}

The chapter will assign parking permits to members living in the house.
If the number of residents with cars exceeds the number of parking spots,
permits will be distributed using a lottery system.
Otherwise, each resident with a car will be given a permit.
In order to receive a permit, members must provide a copy of the vehicle’s
registration to the House Manager.
Members shall not block access to any parking spot.
If a member blocks another car, he must notify the person being blocked in so
that any conflicts can be worked out in advance.
If a car is blocked in and needs to leave, the owner of the blocked car must
notify the owner of the blocking car.
The owner of the blocking car must move it within 8 hours.

\subsection{Non-Resident Member Parking}

If there are any remaining spots after all residents have been assigned
permits, permits may be given to non-resident members using an opt-in lottery
system.
Non-resident members must follow the same rules laid out in
Sub-Section~\ref{sec:resident-parking}.

\subsection{Guest Parking}

If there are any additional parking spots and using them would not affect the
use of parking for Residents and non-Resident Members, a guest may park at the
house with the explicit consent of the House Manager.
The individual responsible for hosting a guest is also responsible for the
vehicle.
The House Manager may request the vehicle be removed from the property at any
time.

\section{Disciplinary Policy}

\subsection{Minor Infractions}

Minor infractions will be dealt with using a 3-strike system.
Examples of minor infractions include, but are not limited to: failure to
complete chores, failure to move a car when needed, and failure to properly
maintain common areas.

1st offense: The House Manager will meet with the responsible member to discuss
house policy.
Additionally, if the infraction was failure to complete a chore, the House
Manager will assign a specific time for the responsible member to complete the
chore and oversee its completion.

2nd offense: The same sanctions as step 1 will be enacted with an additional
warning that the responsible member will be referred to the Kai Cabinet on the
next offense.

3rd offense: The responsible member will be brought to the Kai Cabinet to
discuss the reasons for the offenses.
The Kai Cabinet may impose sanctions as they see fit.

\subsection{Major Infractions}

Major infractions will immediately be brought to the attention of the Kai
Cabinet and the Housing Corporation.
Examples of major infractions include, but are not limited to: possessing
alcohol or illegal drugs in the house, possession of firearms in the house, and
physical altercations.
Major infractions may result in termination of the violator’s lease, in
addition to sanctions determined by the Kai Cabinet.

\section{Housing Selection and Room Selection}

\subsection{Preference for Living in House}

The President, Risk Manager, and House Manager are required to live in the
house.
The order of preference for living in the house is as follows:
\begin{enumerate}
    \item President, Risk Manager, and House Manager
    \item New Members
    \item Sophomores
    \item Juniors
    \item Seniors
    \item Super-seniors
    \item Graduate Students
    \item Alumni
\end{enumerate}

In the event that one group listed above has enough members to exceed the
capacity of any housing options, a lottery for that group will take place to
determine which members shall live in the house.

\subsection{Room Selection}

Room selections will be determined using a lottery system.
Members will be assigned a randomly-generated lottery number from one to the
remaining number of members living in the house.
Members with lower numbers will pick rooms first.
If two members want to room together, they may pair up and use the lower
lottery number.
Additionally, the House Manager may require that certain rooms be designated as
singles, doubles, triples, etc.
based on the number of people living in the house.
Room selection shall occur prior to the end of D-term preceding the lease
period.

\subsection{Reuse of Rooms}

If, after living in the house, a member is selected to live in the house again,
he may choose to remain in his current room or participate in room selection
and choose a new room.
If he chooses to remain in the same room, he will not be assigned a lottery
number and will be guaranteed a spot in his current room.

\section{Summer House Manager}

In the the event that neither the House Manager nor the Risk Manager are able
to remain in the house during the summer and the house will have summer
residents, the House Manager must appoint a Summer House Manager.
The Summer House Manager is expected to live in the house during the summer and
perform the duties of the House Manager until such time as either the House
Manager or Risk Manager can move back into the house.
If a Summer House Manager is not selected and neither the House Manager nor
Risk Manager are able to remain in the house, the house must be closed until
either the House Manager or Risk Manager return.

\section{Equipping of Housing with Chapter Funds}
\label{sec:equipping-of-housing-with-chapter-funds}

No funds from the chapter shall be used for outfitting or equipping any bedroom
or personal area.
Funds from the chapter shall only be used for common spaces, where any member
may benefit (e.g.\ living/common rooms, tech suite, decorations, etc.).
Requests for purchases may be submitted by any initiated member to the Finance
Committee for approval.
Requests must include an item name, item description, vendor, price estimate,
and purpose for benefitting all brothers.
The Finance Committee and Finance Chairman reserve the right to accept or deny
a request based upon its merits and cost.
The House Manager reserves the right to accept or deny a request based on
safety concerns.


\section{Fire Drills}
\label{sec:fire-drills}

The chapter shall hold a fire drill once per term at a date and time to be
specified by the House Manager.
All Residents are expected to attend.
If a Resident cannot attend, they must inform the House Manager at least 24
hours prior to the drill.

\section{Years without a House}
\label{sec:years-without-a-house}

In the event that there exists no property, which is owned or leased by the
chapter or its Housing Corporation, Item~\ref{cha:housing} of these bylaws,
with the exception of Section~\ref{sec:event-cleanup},
Section~\ref{sec:equipping-of-housing-with-chapter-funds}, and this section
(Section~\ref{sec:years-without-a-house}), is suspended for the period of time
that no such property exists.
In the event that a property exists, but no Brother, New Member, Alumnus, or
other associated member of the chapter currently resides on said property, only
Item~\ref{cha:housing} Section~\ref{sec:fire-drills} is suspended, and the
remaining sections of Item~\ref{cha:housing} are to be followed insofar as they
must be, given the lack of residents.

