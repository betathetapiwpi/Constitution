\bylaw{Code of Conduct}

This Code shall be the example by which we, as brothers, commit ourselves for the purpose of becoming Men of Principle.
A brother shall be considered active and in good standing if he has met all of these expectations.
\section{Brotherhood Obligations}
Each brother shall always uphold the Principles and Obligations of the Beta Theta Pi Fraternity.
\section{Financial Obligation}
Each inducted member is required to sign a promissory note, written by the Finance Chairman and approved by the Executive Board, at the end of each semester.
Each brother shall always fulfill his financial obligations within two weeks of being billed; and if not possible, the brother shall work out a payment plan with the Finance Chairman.
Bills shall include, but are not limited to, the following:
Active member local dues
New Member dues
General fraternity dues
Interfraternity council dues
Housing fee
If a brother is not able to meet the financial obligation he has the option of becoming financially inactive as per the Financial Inactivity Bylaw.
\section{Academic Success}
Each brother shall always strive to maintain at least a 2.
 GPA, if a brother is unable to achieve this he will be placed on the Academic Assistance Plan.
\section{Active Participation}
Active brothers are required to attend the following events:
Kickoff and mid-year retreats
Two of the three community service events per term as defined in Section 5
All ritual events
All chapter meetings
All committee meetings
Brotherhood education sessions
Beta Philanthropy events
Beta Fundraising events
In addition, all members of the Executive Board are required to attend all Executive Board meetings.
Excusals and other participation issues will be addressed according to Item 2: Attendance and Accountability Policy.

\section{Community Service}
\subsection{Hours to Complete}
Each active brother shall complete a total of 10 hours per semester.
If a brother is inactive for a term within the semester due to off campus IQP/MQP, co-op, medical leave or other reason determined legitimate by Kai, that brother’s semesterly requirement is reduced by 5 hours per term.
 Brothers are also allowed to complete all of their total semester requirement independent from Beta pending approval from the Community Service Chair.
The Community Service Chair, as outlined in Sub-Section 2, must approve of the service and the brother is responsible for providing the chair with proof of completion.

\subsection{Duties of the Community Service Chairman}
The Community Service Chair will be appointed at least two weeks before the end of each semester by the Programming Committee, and is responsible for organizing community service events throughout each term.
 This brother is required to be on campus all semester and cannot be the President, Programming Chairman, chair of another committee, or officer.
These events include, but are not limited to, the 3 events per term that span at least 2.
 hours each as referred to in Sub-Section 3, as well as working with brothers who cannot attend these events as outlined in Sub-Section 4.
 In addition to this, they are responsible for making sure each brother has completed his required hours, and at their discretion may refer a brother to Kai if their community service hours have not been completed.
\subsection{Community Service Event Planning}
The Community Service Chair and the Programming Committee will be responsible for establishing 3 community service events lasting 2.
 hours or longer during each term.
These events must be established and presented to the rest of the brothers at the start of each term.
At least one of these events must be on a different day of the week than the other two as to accommodate all brothers.

\subsection{Completion of Hours}
Brothers are required to attend 2 of the 3 larger community service events mentioned in Sub-Section 3.
If they cannot do this they must let the Community Service chair know ahead of time so that they can work with them to complete their requirements.

\subsection{Procedure for Incompletion of Service}
If the brother does not meet any one of these requirements then they must follow the procedure outlined in the Attendance and Accountability Policy, listed under Item 2, Section 1.
If the semesters that the brother does not complete his hours are non-consecutive then all three steps outlined in Item 2, Section 1 must be followed through.
If the semesters of non-completion are consecutive then Kai is allowed to vote for Trial by Chapter on the second offense.

\section{Risk Management}
Each brother must follow the Risk Management and Alcohol policies for both this chapter and the general fraternity.

