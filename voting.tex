\bylaw{Voting Procedure}
\section{Voting Methods}
The voting methods for all chapter related activity shall be restricted to the systems listed below.
All methods require quorum
Unanimous Vote
All active and voting brothers must vote to pass the given proposal in order for it to be approved.
Super-Majority Vote
At least three fourths of the active, and voting brothers must vote to pass the given proposal in order for it to be approved.
Simple Majority Vote
More than one half of the active, and voting brothers must vote to pass the given proposal in order for it to be approved.
Instant Runoff Vote
Each active and voting brother ranks a pool of candidates in order of preference.
The votes are tallied, and the candidate with the fewest first choice votes is eliminated.
If two or more candidates are tied for the fewest first choice votes, the eliminated candidate is determined by second choice and so on, until the tie can be broken.
If the candidates remain tied for last, then all such candidates are eliminated.
The votes are re-tallied, with votes for eliminated candidates being counted towards the next-highest choice on that ballot.
This process is repeated until one candidate is the highest remaining choice on more than half of the remaining ballots.
Plurality Vote
The proposal with the most supporting votes from active, and voting brothers will be approved.
\section{Vote Counting}
At chapter with the exception of exec and officer elections, the vote shall be counted by the President and Parliamentarian.
During exec and officer elections the votes shall be counted by the Election Council.
Otherwise the votes shall be counted by the presiding officer.

Recording the vote: All voting results shall be recorded as a pass / fail in the minutes.
\section{Voting Usage}
\begin{itemize}
\item[] Unanimous Vote
\begin{itemize}
\item Initiating new members
\item Recruitment committee bidding of Potential New Members (PNMs).
\item Expelling members
\end{itemize}
\item[] Super-Majority Vote
\begin{itemize}
\item Bidding at chapter of PNMs.
\item Approving financial inactivity
\item Creating new committees
\item Depledging
\item Approving the budget
\item Amending the Constitution or Bylaws
\item Suspending/reinstating Bylaws
\item Suspending/reinstating members
\item Sweethearting
\item Review of an appeal to Kai decision
\item Deciding whether a charge has been proved at a Trial By Chapter
\item Waiving an attendance requirement of a financially inactive member
\item Approving or renewing a temporary emergency dues plan
\item Executive Board decisions within the Executive Board
\end{itemize}
\item[] Simple Majority Vote
\begin{itemize}
\item Kai decisions within the Kai Cabinet
\item Setting penalties at a Trial By Chapter
\item Instant Runoff Vote
\item B-Term Officer Elections
\item Interim Officer Elections
\end{itemize}
\item[] Plurality Vote
\begin{itemize}
\item Kai Elections
\item Creating the Election Council.
\item Electing a convention delegate
\end{itemize}
\end{itemize}

\section{Voting Definitions}
\begin{description}
        \item[Quorum] A quorum shall consist of two-thirds of active-on campus members in good standing for the transaction of all business at any regular meeting of this chapter; however, new members do not count towards or against the quorum in the term of their bidding.
If quorum is not achieved at a regularly-scheduled meeting, voting may not be conducted at that meeting.

\item[Closed Ballot] A system by which the identities of voters are unknown except to those counting the votes.
A Closed Ballot may be collected through a heads down hands up vote, written ballot collection, or an anonymous digital communication platform such as email.
The method of ballot collection shall be at the discretion of the presiding officer.

\item[Abstention] An abstaining brother shall not be considered a voting brother.
\end{description}
\section{Voting Rights}
All active brothers and new members who have been inducted are permitted to vote, unless otherwise specified by the Constitution or Bylaws.

