\chapter{The New Member Process}
\label{cha:the-new-member-process}

\section{New Member Periods}

There shall be no more than four new member periods per year: one every term.
The new member period shall last eight weeks at maximum and may not be extended
except by approval of the General Secretary and the Office of Fraternity and
Sorority Life under extenuating circumstances (e.g., illness or a death in the
family).

\section{New Members and Letters}

New members of the Fraternity shall not wear the Greek letters until they learn
the meaning of the letters during the Initiation Ceremony.
However, new members may wear any other apparel with the crest or the English
words “Beta Theta Pi”.

\section{New Members' Voting Rights}

New members will be allowed to vote on all matters of business (except for
matters relating to Ritual) after they have been inducted.

\section{New Member Requirements}

To become a member, a new member must complete the following criteria:
\begin{itemize}
    \item Gain the approval of the active brothers of the chapter, with respect
        to his embodiment of the Values of Beta Theta Pi and his integrity to
        adhere to the Obligations
    \item Obtain a cumulative university GPA of at least 2.8. If the new member
        is an entering freshman, he must have placed in the top twenty percent
        of his high school graduating class or obtain a high school GPA of at
        least 3.3.
    \item Complete the chapter’s New Member Education program to the
        satisfaction of the Education Committee.
\end{itemize}

\section{New Member Review Sessions}

A new member can be depledged at one of two New Member Review sessions that
shall be held throughout the New Member program by the members during regularly
scheduled meetings.
The first review session shall be held at the midway point of the New Member
Education process.
The final review session coincides with initiation voting.
Voting to depledge is carried out according to the voting procedure.

\section{New Member Initiation Requirements}

Before initiation into the chapter, a new member must do the following:
\begin{itemize}
    \item Complete the New Member Education Program.
    \item Have all fraternity bills, including the initiation fee, submitted
        and paid to the satisfaction of the Finance Chairman.
    \item Meet the standards outlined by the General Fraternity.
\end{itemize}

\section{Roll Number Assignment}

Roll numbers will be assigned to new members being initiated in alphabetical
order by last name.

\section{Post-initiation Training}

All newly initiated brothers must participate in a post-initiation training,
which will review the items outlined in Chapter XIV of the Ritual of Beta Theta
Pi.
Ideally, this training will happen within one week of initiation

\section{Alcohol Policy}

Newly initiated brothers must explicitly learn about the expectations regarding
alcohol and other drugs, as outlined in the Risk Management policy of the
general fraternity

\section{Post-Initiation Competency}

In accordance with the Ritual of Beta Theta Pi, brothers will be considered
competent who can demonstrate their mastery of the Ritual.
Mastery is defined as intimate familiarity with the items outlined in Chapter
XIV of the Ritual of Beta Theta Pi such that:
\begin{itemize}
    \item The brother must demonstrate their knowledge, in person, to the
        Brotherhood Chairman or a ritually competent brother with the approval
        of the Brotherhood Chairman.
    \item Once a brother is deemed competent he shall be considered competent
        indefinitely.
\end{itemize}

