\chapter{Financial Operations}
\label{cha:financial-operations-bylaws}

\section{Price of Dues}

Dues will be set at \$500 per semester per person for spring 2018 dues, not
including fees from Billhighway.
In the fall semester of every year that is divisible by three (e.g. 2019, 2022,
2025, etc.), the Finance Committee will review this amount and adjust it solely
to account for any and all inflation that has occurred in the U.S. economy
since the last adjustment.
The price of dues shall remain constant unless changed in this way or by a
temporary emergency dues plan (see Sub-Section~\ref{sec:emergency-adjustments}).

\subsection{Emergency Adjustments to the Price of Dues}
\label{sec:emergency-adjustments}

The Finance Committee can draft a temporary emergency dues plan if the price of
dues needs to be changed because of an unforeseen emergency situation.
This plan must be discussed by the Executive Board and brought before the
chapter for discussion and voting.
The plan must be approved by the chapter as per the voting procedure to take
effect.
Temporary emergency dues plans may only be in effect for one semester, and must
be renewed by the chapter as outlined by the voting procedure at the end of
each semester.

\section{The Budget}
\label{sec:the-budget}

In the fall semester of 2017, and each subsequent year that is divisible by
three, the Finance Committee will assign a fall percentage and a spring
percentage to each of the following categories, such that the sum of all the
percentages for each semester (fall or spring) is 100\%:
\begin{itemize}
    \item President 
    \item Brotherhood Committee
    \item Communication Committee
    \item Finance Committee
    \item Programming Committee
    \item Education Committee
    \item Risk Management Committee (including housing)
    \item Recruitment Committee
    \item Savings
\end{itemize}
The designated percentage of the chapter’s total income
from dues for each semester will be allocated to the category.
These percentages will be presented to the Executive Board by the Finance
Chairman at or before the last Executive Board meeting of the first term of
each semester (A and C terms).

\subsection{Drafting the Budget}

The President and the Chairmen of each Committee shall present a proposed plan
for how to use the funds allocated to them or their Committee to the Executive
Board by the second Executive Board meeting of the second term of each semester
(B and D terms).

\subsection{Approving the Budget}

After discussion by the Executive Board, all of the plans proposed for the use
of allocated funds (the budget) will be presented to the chapter and voted on
as per the voting procedure.
In B term, this vote shall occur after Induction.
If the budget does not pass, the Finance Committee and Executive Board will
revise the budget until it is passed by the chapter.

\subsection{Adjustment to the Budget}

If an Executive Officer wants the percentages allocated to each category to be
changed between the ordinary re-evaluations in each year divisible by three
(see Item~\ref{cha:financial-operations}, Section~\ref{sec:the-budget}), he may
present an alternative percentage distribution to the Finance Committee.
The Finance Committee will approve or deny the alternative percentage
distribution within one week.
If it is approved, the alternative percentage distribution will be discussed by
the Executive Board and presented to the chapter with the next semester’s
budget.
It will be approved or denied by the chapter as a part of the next semester’s
budget.

\subsection{Prohibited Uses of Budgeted Funds}

Funds allocated in the manner described in this section shall not be used for
any Prohibited Use unless either
\begin{enumerate*}[label={\alph*)}]
    \item such use is clearly denoted in the budget
approved by the chapter and this Sub-Section was specifically referenced when
the budget was debated by the chapter, as reflected in the minutes; or \item such
use is approved by a super-majority vote of the chapter after the budget is
passed.
\end{enumerate*}
The Prohibited Uses are: 
\begin{itemize}
    \item donation of the funds
\end{itemize}

\section{Reserve Funds}

Each committee, with the exception of Risk Management, shall have a reserve
fund in order to offset any unexpected costs.
The Executive Board may decide via the voting procedure to add money to
individual Reserve Funds from the chapter’s General Savings Fund.
In addition, any money allocated to a Committee but not spent by the end of the
semester will be automatically added to that Committee’s Reserve Fund.
The amount of money in each Reserve Fund can never exceed \$250; any money that
would be deposited in a Reserve Fund past \$250 will instead be put into the
chapter’s General Savings Fund.

\section{General Savings}

For money to be taken out of the chapter’s General Savings Fund, the proposed
expenditure must be presented to, and approved by the Finance Committee.
If the President or Finance Chairman determine the situation to be an
emergency, and the Finance Committee will not meet before the expenditure must
be made, the Finance Chairman and President may decide to approve or deny the
expenditure independent of the Finance Committee.
If this occurs, the expenditure must be disclosed to the Finance Committee as
soon as possible.

\section{Reimbursements}

Reimbursements for individual brothers are at the discretion of the Finance
Chairman.
Reimbursements may be distributed by check or Venmo.
The Finance Chairman shall make his best effort to distribute approved
reimbursements within one week of their request.

\subsection{Reimbursement Denial Appeals}

If a brother believes he deserves a reimbursement for a legitimate expenditure,
and the Finance Chairman denies him a reimbursement, he may appeal the denial.
The brother may appeal to the Kai Cabinet.
The Kai Cabinet shall hear the brother’s case, including input from the Finance
Committee, and vote to determine whether the brother deserves a reimbursement.
If the reimbursement is approved by the Kai Cabinet, the Finance Chairman shall
give the reimbursement to the brother within one week.

\section{Venmo}

The official Venmo account of the Eta Tau chapter of Beta Theta Pi shall be
\texttt{@BetaThetaPiWPI}.
Venmo may be used to collect payments other than dues from brothers or to
distribute reimbursements.
Any money in the official Venmo account must be transferred to Billhighway by
the end of each semester.
The Finance Chairman and President shall have control of the chapter Venmo
account, and may allow other brothers to temporarily have access to the chapter
Venmo account at their discretion.

