\chapter{Elections}
\label{cha:elections}

\section{B-Term Officer Elections}
\label{sec:b-term-officer-elections}

Elections for all executive officers and other officers will be held every
B-Term in two parts: executive officer elections and other officer elections

\subsection{Timing}
\label{sec:timing}

Nominations for executive officer positions should be opened four weeks prior to
the end of B-Term.
One week later, the nominated brothers will accept or decline
their nominations, and nominations for other officer positions should be opened.
The following week the executive officer election will take place at a regularly
scheduled chapter meeting, and brothers will accept or decline their nominations
for other officer positions.
The following week the election for other officers will take place at a
regularly scheduled chapter meeting.
All officers shall serve for one year, beginning the next term after their
Election, with the exception of House Manager (see
Item~\ref{cha:the-new-executive-board}, Section~\ref{sec:entering-office}).

\subsection{Nomination for Election}
\label{sec:nomination-for-election}

A brother may be nominated for any number of positions, but may only accept
nominations for up to two positions.
In order to accept a nomination, a brother must submit a written or recorded
speech for that position to the election council.

\subsection{Nomination Period}
\label{nomination-period}

After nominations for positions close and all speeches have been submitted, the
election council will publish all speeches for viewing by the chapter.
For each position being elected, a Slack channel will be made with the purpose
of allowing brothers to ask questions of the nominees.
The discussions in these channels will be moderated by the election council.

\label{Election Procedure}
\label{election-procedure}

Elections for executive officer positions are to proceed in the following order:
President, Brotherhood Chairman, Communications Chairman, Education Chairman,
Finance Chairman, Programming Chairman, Recruitment Chairman, Risk Management
Chairman.
Elections for other officer elections are to proceed in the following order:
House Manager, Parliamentarian, Chorister, Historian.
For each position, the candidates will read their speeches that were submitted
the previous week.
The election council will cut off speeches after 2 minutes for all positions
except President, which will have 5 minutes.

After all speeches are given for a single position, candidates will be asked to
leave the room, and discussion will be opened.
The election council will regulate these discussions to prevent circular
arguments.
After 5 minutes, the council should regulate more heavily, to encourage brevity.
If a question comes up during the discussion, the candidate may be called in to
answer it briefly.
When the election council determines that the discussion has come to a close,
voting will commence.
A brother may also motion to vote.
Once voting commences, candidates will be asked to re-enter the room so that
they can cast their votes.

If a position is not filled then the election procedure for the unfilled
positions will be an interim election, as outlined in Item~\ref{cha:elections}
Section~\ref{sec:interim-elections}.
The position will not be considered an interim position.

\subsection{Election Council}
\label{sec:election-council}

The Election Council is responsible for counting the ballots of an election in
an honest manner representative of their membership in the Beta Theta Pi
Fraternity.
It shall be composed of three initiated brothers who voluntarily choose not to
allow their names to be put forth for nomination to elected positions.

The Election Council shall be chosen at a regularly scheduled chapter meeting at
least one week before the start of the Executive Board election process.
If more than three brothers volunteer to be on the Election Council, the
composition of the Election Council shall be decided per the voting procedure
item.

The Council shall compile all ballots immediately after voting ends and create
the new Executive Board based on the results.
The slate for the Executive Board shall be presented to the whole chapter as
soon as possible.

If the election results in a tie, the Election Council removes all other
candidates and reopens discussion, then holds another vote.
If there is again a tie, the Election Council will remove one candidate, and the
chapter will again discuss and vote on the remaining candidates.
This process repeats until one candidate is left.

\section{Interim Elections}
\label{sec:interim-elections}

Elections for interim positions should be conducted at a time decided by the
President to best allow for a smooth transition.
The nominations for interim positions will occur at least one week prior to the
election.
Interims shall serve for a period of time decided by the President and not
exceeding one year.

\section{Kai Cabinet Elections}
\label{sec:kai-cabinet-election-procedure}

Nominations for the Kai Cabinet should take place during the second to last week
before the end of each semester.
The election will take place the following week.
Kai Cabinet members shall serve for one semester following their election.

\section{Qualifications for Election}
\label{sec:qualifications-for-election}

All elected brothers must be in good standing to hold office.
In addition, all brothers running for executive positions must demonstrate
ritual competency before they may be elected.
The ritual competency requirement may be waived if the person running is a New
Member who is scheduled for initiation before the end of the current academic
term, or a brother who is currently away.
An away brother must demonstrate their ritual competency within a week of the
start of the next term or a meeting with the Kai Committee will be scheduled.
A brother cannot hold two officer positions simultaneously; if he is elected to
one position, he is immediately ineligible to run for any other officer
positions with the same term or overlapping terms with that position.
An officer also cannot serve on the Kai cabinet.

