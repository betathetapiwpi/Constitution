\chapter{Discipline of Members}
\label{cha:discipline-of-members}

\section{Jurisdiction of Discipline}

The chapter shall have the power to discipline a member of the Fraternity who
is not in good standing.
The chapter will have jurisdiction over all members who are on the rolls as
affiliated to the chapter, whatever their status, and overall undergraduate and
graduate school members of the Fraternity’s chapter.

\section{Disciplinary Method}

The discipline of a member by a chapter will be conducted using either the
Informal or the Formal Procedure as herein provided.

\section{Erroneous Discipline}
If requested by the Accused and if it appears that an error has been made or
that the interests of justice so required, the chapter may vote its decision
according to the voting procedure and may amend, modify or vacate that final
decision provided: it does so within one year from the date the decision became
final and the decision was not the expulsion of a member from the Fraternity.

\section{Informal Procedure}

The Informal Procedure for the Discipline of Members

\subsection{Application of Informal Procedure}

The Informal Procedure shall be used by the Kai Cabinet when the performance of
a member is questioned.

\subsection{Informal Procedure Participants}

The Brotherhood Chairman shall chair the Informal Procedure hearing.
Any decision reached by the Kai Cabinet must be approved by a simple majority
vote of the Kai Cabinet.
In the event that the vote is split, then and only then must the Brotherhood
Chairman vote on the final decision.

\subsection{Exceptions}
\label{sec:exceptions}
The Informal Procedure may be used in all cases except the following:
\begin{enumerate}
    \item In cases involving the possible expulsion of the Accused or
        suspension of the Accused for more than one year.
    \item In cases in which the Accused is no longer in school or his
        whereabouts are unknown.
    \item In cases in which the Accused requests to be tried under the formal
        procedure established below.
\end{enumerate}

\subsection{Rights for the Accused}

The Accused has the right to be present at the hearing, hear the evidence
against him, ask questions of various witnesses, and present evidence in his
own behalf.
The Accused may also have a member of the Fraternity appear on his behalf to
help him in his defense.
The procedure shall provide a way for the Accused and the chapter to enter into
an agreement to settle the charge in a way agreeable to both sides.
If no compromise between the Kai Cabinet and the Accused can be agreed upon,
the Kai Cabinet shall recommend the Formal Procedure, as outlined in
Item~\ref{cha:discipline-of-members} Section~\ref{sec:trial-by-chapter}: The
Formal Procedure for the Discipline of Members: Trial by Chapter.

\subsection{Potential Penalties}

The Kai Cabinet may impose one or more of the following:
\begin{itemize}
    \item A warning
    \item Censure (either in private or before the chapter)
    \item A written apology to those wronged
    \item Loss of social privileges
    \item Loss of access to or position in room draw for chapter house
    \item Exclusion from various fraternity activities
    \item Removal from office/position
    \item Payment for loss or damage
    \item Loss of voice or voting privileges in chapter meetings
    \item Other penalties agreed upon by the Kai Cabinet and the Accused
    \item Suspension of membership in the Fraternity for a period not to exceed
        one year
    \item A recommendation to the chapter for a Trial by Chapter
\end{itemize}

\subsection{Administrative Suspension}

In the event the Accused is suspended from the Fraternity for more than thirty
days, notice of the suspension must be sent to the Administrative Secretary
within ten days after the penalty is imposed.
No other result needs to be reported.

\section{Formal Procedure: Trial by Chapter}
\label{sec:trial-by-chapter}

The Formal Procedure for the Discipline of Members: Trial by Chapter:

\subsection{Application of Formal Procedure}

The Formal Procedure shall be used to allow the chapter as a whole to
discipline its own members.

\subsection{Restrictions on Formal Procedure}

The Formal Procedure may only be used when the Informal Procedure cannot be
used for the reasons set forth in Item~\ref{cha:discipline-of-members}
Sub-Section~\ref{sec:exceptions} of these Bylaws.

\subsection{Charges}

The chapter must prepare the charges to be filed and forward those charges to
the District Chief or Regional Chief for review in advance of presenting them
to the accused.
The chapter leadership will make arrangements for the District Chief or
Regional Chief to host a call with the executive committee and those bringing
charges to them as soon as practical.
The District Chief or Regional Chief will counsel the members on the importance
of accuracy, validity, and thoroughness of the charges, and other procedural
issues worthy of their consideration.
The chapter will then present the charges to the accused.

\subsection{Chapter Procedure when a Brother is Accused}

Due Notice, as defined in the Appendix, must be given to the Accused.
At the time set for trial, the chapter, with a quorum present, must hear the
charge against the Accused, decide whether the charge is true and fix the
penalty upon a finding of guilt.
The Accused may waive their right to such hearing by voluntarily admitting to
the charges.

\subsection{Role of the Kai Committee:} %% Remove this colon

The Kai Committee shall organize and present the case against the Accused.

\subsection{Trial by Chapter Chair}

The parliamentarian shall chair the Trial by Chapter.
If the parliamentarian is not, as per his discretion, willing to serve as the
chair, the chapter may approve another chair by simple-majority vote.

\subsection{Admission of Charge}

The Accused shall be given an opportunity to admit or deny the charges.
If the Accused does not attend the hearing, after being given personal service
of notice, the chapter may consider the failure to appear as an admission of
the charges.
If the charge is admitted, the chapter will then, by a simple majority vote,
set the penalty.

\subsection{Order of Decision}

If the Accused denies the charge, the chapter will consider the evidence and
arguments presented by each side, and will then vote on this question: Has the
charge been proved? If a super-majority of the chapter votes in the
affirmative, the Accused stands convicted, and the chapter will then consider
and decide the penalty to be imposed by majority vote.
If not, the Accused remains not guilty.

\subsection{Rights of the Accused}

The Accused has the following rights:
\begin{itemize}
    \item The right to due notice as defined below.
    \item The right at trial to be heard, to present evidence, to hear the
        evidence against him, and to be represented by an advisor or counselor
        who shall be a member of the Fraternity.
    \item The right to be present throughout the proceedings except when the
        chapter makes its decision.
    \item The right to appeal as provided herein.
    \item The Accused may waive some or all of his rights.
\end{itemize}

\subsection{Potential Penalties}

The chapter may impose one or more of, but not limited to, the following
penalties:
\begin{itemize}
    \item A warning
    \item Censure before the chapter
    \item A written apology to those wronged
    \item Loss of social privileges
    \item Exclusion from various fraternity activities
    \item Fines
    \item Payment for loss or damage
    \item Other penalties as might be agreed upon by the chapter and the
        Accused
    \item Suspension of membership in the Fraternity for a period not to exceed
        four years
    \item Expulsion from the Fraternity
\end{itemize}

\subsection{Minutes of the Trial}

Minutes shall be kept of the proceedings of the chapter summarizing the
evidence presented and including:
\begin{itemize}
    \item Date of Hearing;
    \item Name and Address of accused;
    \item Chapter Name and Address;
    \item Name of the chair of Trial;
    \item Roll number of accused;
    \item The written charges;
    \item The specific conduct, found by the Chapter to be true and to
        constitute a violation of the Code or Bylaws;
    \item The decision of the chapter;
    \item The penalty imposed;
    \item The name and address of the person who kept the minutes on behalf of the
    chapter.
\end{itemize}

A copy of the Minutes must be provided to the District Chief, Executive
Director, and the accused within 10 days of the trial.


\section{Rules of Discipline Appeals}

Appeals to the decision of the discipline of members:

\subsection{Informal Procedure Appeals}

The Informal Procedure for the discipline of members may not be appealed.

\subsection{Formal Procedure Appeals}

The Formal Procedure for the discipline of members may be appealed when:
The Accused may appeal either the finding of guilt, the penalty imposed, or
both.
The appeal is first to the Board of Trustees and then to the next General
Convention.
To appeal, the Accused must send a letter to the Administrative Secretary,
within the time limits herein provided, asking that the chapter decision be
reviewed and stating the grounds for review.
The Administrative Secretary will direct the appeal to the Board of Trustees.
If the Accused was personally given notice, he will have ten days from the day
of trial to appeal the decision of the chapter.
If the Accused was not personally given notice, he will have 90 days from the
date the Administrative Secretary receives a copy of the Minutes in which to
appeal.
A decision is considered final once the time for appeal expires or the next
General Convention has concluded its review, whichever comes first.

